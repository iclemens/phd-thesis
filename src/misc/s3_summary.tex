\clearpage
\pagestyle{empty}

\chapter*{Nederlandse samenvatting}
\phantomsection
\addcontentsline{toc}{chapter}{Nederlandse samenvatting}

Wanneer men door de omgeving navigeert, dan veranderen de visuele, auditieve, vestibulaire, tactiele, en motorische signalen die de hersenen bereiken. Ondanks deze, door beweging geinduceerde, veranderingen van de signalen, nemen we de wereld als een stabiele werkelijkheid waar. We onderhouden een gevoel van waar we zijn, wat onze orientatie in de wereld is, en interacteren ogenschijnlijk moeiteloos met de objecten om ons heen. De werking van deze processen is het onderwerp van dit proefschrift. Met behulp van computationele modellen en grondig psychometrische tests leggen we de fysische en biologische beperkingen in de interactie tussen het evenwichtsorgaan en andere sensorische systemen voor het waarnemen van ruimtelijke orientatie en zelf-beweging bloot. Dit resulteerde in de volgende bijdragen aan het veld:

\begin{description}
\item[\Hoofdstuk{p1}] Statistische optimale integratie kan de manier waarop het lichaam somatosensorische, proprioceptieve, en vestibulaire signalen integreert voor het waarnemen van spatiele orientatie goed beschrijven. Hoewel deze sensorische systemen moeilijk afzonderlijk te bestuderen zijn, hebben we de statistische eigenschappen bloot kunnen leggen door statische optimaliteit als uitgangspunt te nemen. De gevonden eigenschappen hebben we vervolgens gekoppeld aan klinische afwijkingen in twee patientgroepen.
\item[\Hoofdstuk {p3}] Oculomotorische signalen beinvloeden de waarneming van zelf-beweging, zelfs wanneer deze oogbewegingen in strijd zijn met de vestibulaire informatie en in afwezigheid van visuele stimulatie.
\item[\Hoofdstuk {p4}] Deze oculomotorische signalen zijn slechts een rudimentaire bron voor de waarneming van zelf-beweging: hoewel de oogbewegingen schalen met de fixatie afstand, wordt dit effect niet voldoende gecompenseerd.
\item[\Hoofdstuk {p2}] Zelf-bewegings signalen beinvloeden de dynamische perceptie van de wereld om ons heen. Fouten die ontstaan in dit process wekken de suggestie dat oog-centrische referentie kaders gebruikt worden voor de onderliggende berekeningen.
\end{description}

In de volgende paragrafen, zullen we elk van deze bijdragen in meer detail bespreken.

\section{Multisensorische verwerking in ruimtelijke orientatie}
Vele studies hebben aangetoond dat de hersenen de signalen van verschillende zintuigen op een statisch optimale wijze integreren voor verdere verwerking. Normaliter wordt statische optimale integratie aangetoond doordat de afzonderlijke ruisniveaus van elke modaliteit kunnen voorspellen wat het ruisniveau van alle modaliteiten in combinatie is. Het is het moeilijk om zintuiglijke signalen afzonderlijk te meten in ruimtelijke oriëntatie kijken, met name omdat het vestibular systeem niet of moeilijk uit te schakelen is bij het meten van somatosensorische bijdragen.
In \Hoofdstuk{p1} hebben we statistische optimaliteit als uitgangspunt genomen, en op die manier geprobeerd de perceptuele verschillen in de waarneming van het hoofd in de ruimte ten opzichte van het lichaam in de ruimte te verklaren. Met behulp van een psychofyische aanpak hebben we zowel de waarneming van lichaamshoek (subjective body tilt; SBT) alsmede de waarneming van de visuele verticaal (subjective visual vertical; SVV) in zeven proefpersonen gemeten. Omdat zowel de SVV als de SBT gebruik maken van dezelfde sensorische signalen, waren we instaat om een model met 7-parameters te fitten op de verkregen data. Een van de parameters was de ruis op de nek proprioceptie. Omdat de ruis op nek proproceptie wel in afzondering te meten is, hebben we de model waarden kunnen bevestigen. Daarnaast hebben we het model gevalideerd door te laten zien dat voorspellingen over vestibulaire en somatosensorische patienten in overeenstemming zijn met gepubliceerde gegevens. We concluderen dat een Bayesiaanse aanpak de typische taak-afhankelijke verschillen in de waarneming van ruimtelijke orientatie goed kan verklaren. In een vervolg op dit werk werd deze aanpak onlangs toegepast op een patientenpopulatie met verlies van het complete vestibulaire systeem \cite{alberts2015}. Deze patienten voerden de taak vergelijkbaar met de controle populatie uit, wat de suggestie wekt dat de sensorische gewichten verschoven zijn van vestibulaire naar somatische sensoren.  


\section{Oogbewegingen beinvloeden de waarneming van zelf-beweging}
Tijdens zelf-beweging maken we meestal oogbewegingen om te compenseren voor verstoringen op het netvlies. In \Hoofdstuk{p3} onderzochten we of deze oogbewegingen ook een omgekeerde rol hebben, en gebruikt worden als bron signaal voor zelf-beweging. Om een antwoord op deze vraag te geven, vroegen we deelnemers om de bewogen afstand te vergelijken in twee opeenvolgendepassieve zijwaartse verplaatsingen. De ogen keken tijdens deze verplaatsingen naar een wereld- of lichaams-vast doel. Onze resultaten laten zien dat bewegingen als korter werden waargenomen wanneer naar een lichaams-vast doel werd gekeken, en de oogbewegingen dus werden onderdrukt. Met behulp van een lineair model hebben we de relatieve bijdrage van het op oogbewegingen gebaseerde  verplaatsingssignaal ten opzichte van de verstibulaire bijdrage. De oogbewegingen zijn verantwoordelijk voor ongeveer 25 procent van de in ons experiment waargenomen beweging. Het model is onafhankelijk gevalideerd doordat we de effecten van oogbewegingen op de waarneming van zelf-beweging konden voorspellen wanneer er geen fixatie punt werd aangeboden tijdens de verplaatsing. Dit laat zien dat oogbewegingen zelfs gebruik worden voor het waarnemen van zelf-beweging zonder visuele stimulatie, en wanneer de oculomotorische en vestibulaire schattingen met elkaar in conflict zijn, zoals tijdens lichaams-vaste fixatie. We veronderstellen dat de nadelige gevolgen van deze schijnbaar rigide opstelling minimaal zijn onder natuurlijke omstandigheden, omdat de oogbewegingen en zelf-bewegingen normaliter sterk gecorreleerd zijn. Daarnaast zorgt zelf-beweging voor verplaatsing van de beelden op de retinae, deze verplaatsingen worden ook gebruikt voor zelf-beweging, wat de relatieve bijdrage van de oogbewegingssignalen beperkt. 

\section{Gedeeltelijke compensatie voor fixatie diepte in de waarneming van zelf-beweging}
Als oogbewegingen inderdaad gebruikt worden voor de waarneming van zelf-beweging, dan moeten de hersenen deze oogbewegingen corrigeren voor de fixatie diepte om tot een correcte waarneming te komen. Dit komt doordat de amplitude van de oogbewegingen, bij gelijkblijvende afstand, afhangt van de fixatie diepte: als men bijvoorbeeld in de verte kijkt dan bewegen de ogen nauwlijks vegelijken met wanneer men dichtbij fixeert. In \Hoofdstuk{p4} hebben we onderzocht of de hersenen fixatie diepte inderdaad meenemen wanneer zij oogbewegingen gebruiken voor de waarneming van zelf-beweging. In een 2-AFC taak vergelijkbaar met die gebruikt is in \Hoofdstuk{p3} moesten deelnemers aangeven of de tweede van twee opeenvolgende passieve bewegingen langer of korter was dan de eeste. Tijdens deze bewegingen moeten de proefpersonen naar een dichtbij of veraf doel fixeren, dat vast stond ten opzichte van de wereld of het lichaam. Onze resultaten laten zien dat bewegingen als korter werden waargenomen naarmate wereldvaste fixatiepunten dichter kwamen te liggen. Er is echter geen effect van diepte wanneer we naar een lichaamsvast fixatiepunt kijken. Uit deze bevindingen kunnen we concluderen dat oogbewegingen slechts rudimentaire informatie bieden over zelf-beweging, waarbij de compensatie voor fixatie diepte op zijn best partieel is.

\section{Oog-centrische effecten in het bijwerken van spatiele informatie}
Deze zelf-bewegingssignalen worden onder andere gebruikt om ego-centrische representaties van de omgeving bij te werken. In \Hoofdstuk{p2} onderzoeken we hoe, en in welke mate, de hersenen de verschillende zelf-bewegingssignalen integreren voor om deze representaties bij te werken. Proefpersonen werden zijwaarts heen-en-weer gewogen terwijl ze een doel fixeerden dat vast stond in de wereld. Wanneer de bewegingsrichting veranderde, werd een lampje voor of achter het fixatiepunt geflitst. De proefpersonen werd gevraagd de positie van dit lampje te onthouden. De relatieve positie van de proefpersoon tenopzichte van het lampje veranderd echter wanneer hij van de ene naar de andere kant beweegt. Een halve cyclus na de eerste flits, hebben we getest of deze beweging correct is meegenomen door een ander lampje links of recht van het onthouden lampjete flitsen. De resultaten laten zien dat zowel de richting als de grootte van de correctiefout afhangt van de locatie van het object relatief tenopzichte van het fixatie punt, dit suggereert dat het object relatief tenopzichte van het fixatiepunt wordt opgeslagen door de hersenen. Verder laten we zien dat de fout kan worden verklaard door zowel een onderschatting van de bewogen afstand, alsmede door een bias van de visuele objecten richting de fovea.

\section{Slotopmerkingen}
De experimentele hoofdstukken in dit proefschrift hebben de ruimtelijke waarnemen van proefpersonen als gemeenschappelijk thema. In \Hoofdstuk{p1} hebben een statisch optimaal integratiemodel gebruikt om zowel de accuraatheit als de precisie van deze ruimtelijke waarnemingen verklaren. In dit spport modellen worden de verschillende sensorische signalen gewogen aan de hand van de onzekerheid in het signaal. In de hoofdstukken over de waarneming van zijwaartse verplaatsing (\Hoofdstuk{p3,p4}) hebben we ook gebruik gemaakt van een gewogen combinatie van sensorische signalen, maar deze zijn niet gebaseerd op de onzekerheid. Idealiter zou men dezelfde insteek gebruiken als in \Hoofdstuk{p1} om de resultaten uit deze hoofdstukken te analyseren. We hebben de eerste stappen genomen om tot zo een model te komen, maar er lijken tenminste twee grote uitdagingen die dit bemoeilijken. Ten eerste vraagt de geometrie van de zelf-bewegingsexperimenten om het gebruik van scheve kansverdelingen, dit maakt dat standaard tecnieken die gebruikt worden bij normale kansverdelingen onbruikbaar, waardoor we aangewezen zijn op bijvoorbeeld sequentiele Monte Carlo methoden. Daarnaast is de dataset te beperkt om alle vrije parameters die een kansmodel met zich mee zou brengen op een consistente manier te schatten. Het ontwikkelen van een compleet optimaal integratiemodel voor het waarnemen van zelf-beweging is een logische volgende stap in dit onderzoek.

In \Hoofdstuk{p2} laten we zien dat oogbewegingen die veroorzaakt worden door het fixeren van een wereld-vast fixatiepunt gebruikt worden om de herrinerde locatie van doelen in de omgeving bij te werken. Als het onderliggende signaal dat gebruikt wordt voor de waarneming van zelf-beweging (zie \Hoofdstuk{p3,p4}) ook gebruikt wordt voor het bijwerken van onthouden spatiele doelen, dan zou het soort fixatiepunt (wereld- of lichaams-vast) een effect moeten hebben op de fout in het bijwerken van deze doelen. Resultaten van een nieuw onderzoek suggereren dat het soort fixatie inderdaad een dergelijk effect laat zien (\cite{clemens2010}). We lieten proefpersonen na een zijwaartse beweging rijken naar onthouden doelen. De eerste resultaten laten zien dat fouten in rijkbewegingen groot zijn wanneer een lichaams-vast doel wordt gefixeerd, terwijl er geen fouten optreden bij de fixatie van een wereld-vast doel (\cite{clemens2010}). De voorbarige conclusie is dat oog-bewegingen inderdaad gebruikt worden om zelf-beweging te schatten voor het bijwerken van onthouden doelen.

Tot slot hoop ik met deze these een bijdrage geleverd te hebben aan de vooruitgang in het begrip van de mechanismen achter ruimtelijke orientatie en de waarneming van zelf-bewegingen. Natuurlijk laat deze these veel vragen liggen voor vervolgonderzoek, en roept het zelfs nieuwe vragen op. Deze onderzoeken zouden niet alleen naar de computationele en theoretische mechanismen moeten kijken, maar ook de neurale implementatie van deze mechanismen in het brein mee moeten nemen.

