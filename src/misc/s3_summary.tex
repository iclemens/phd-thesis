\clearpage
\pagestyle{empty}

\chapter*{Nederlandse samenvatting}
\phantomsection
\addcontentsline{toc}{chapter}{Nederlandse samenvatting}

Navigeren door de omgeving oproept complexe veranderingen van visuele, auditieve, vestibulaire, tactiele en
motor inputs naar de hersenen. Toch, ondanks deze beweging veroorzaakte veranderingen van de input, we de wereld waarnemen als een stabiele werkelijkheid, maar ook een geïntegreerde gevoel van waar we zijn, hoe we zijn gericht, en in staat zijn op te sporen en te handelen in plaats moeiteloos op omliggende objecten. Hoe dit vermogen komt is ook het onderwerp van dit proefschrift. Het doel van het onderzoek beschreven in dit proefschrift was om computationele modellen op te bouwen en uit te voeren grondige psychometrische tests om de fysische en biologische beperkingen te ontrafelen op de interactie tussen het evenwichtsorgaan en andere sensorische systemen voor ruimtelijke oriëntatie en zelf-waarneming van beweging. Dit resulteerde in de volgende bijdragen aan het veld:

\ Begin {description}
\ Item [\ hoofdstuk {p1}] Statistische optimaliteit kan goed zijn voor de manier waarop het lichaam somatosensorische, zijn nek proprioceptieve en vestibulaire signalen geïntegreerd in ruimtelijke oriëntatie waarneming. Hoewel deze sensoren niet afzonderlijk kan worden onderzocht door optimaliteit als uitgangspunt geluidsemissie eigenschappen kunnen worden bepaald en gekoppeld aan klinische afwijkingen die worden gezien name patiëntengroepen.
\ Item [\ hoofdstuk {p3}] oogmotorische signalen beïnvloeden eigen waarneming van beweging, zelfs in afwezigheid van optic flow of andere visuele stimulatie, en zelfs in strijd met vestibulaire informatie.
\ Item [\ hoofdstuk {p4}] Dit oculomotorische signalen moet zo rudimentair cue om zelf-waarneming van beweging worden beschouwd; Het is niet veridically geschaald fixatie afstand in de perceptie van lichaam vertaling.
\ Item [\ hoofdstuk {p2}] Self-motion signalen interactie met de dynamische perceptie van de buitenwereld. Fouten die zich voordoen bij deze werkwijze suggereren het gebruik van een blik gecentreerd referentieframes in de onderliggende berekeningen.
\ End {description}

In de volgende paragrafen zullen we een gedetailleerd overzicht van elk resultaat te bieden.

\ Section {Multisensorische verwerking in ruimtelijke oriëntatie}
Vele studies hebben aangetoond dat de hersenen combineert luidruchtige sensorische signalen in een statistisch optimale wijze. Dit gebeurt normaal door te laten zien dat het geluidsniveau van elke afzonderlijke modaliteit prestaties in combinatie kunnen voorspellen. In ruimtelijke oriëntatie is het moeilijk om zintuiglijke signalen afzonderlijk te meten als men niet kan uitschakelen het vestibulair gevoel bij het meten van somatosensorische bijdragen. In \ hoofdstuk {p1} daarom gebruikten we een statistisch optimale integratie-model als uitgangspunt, en geprobeerd om rekening te houden met de perceptuele verschillen gevonden wanneer indringende kop-in-ruimte versus body-in-ruimte oriëntatie. Met een psychometrische benadering, testten we zowel de perceptie van lichaam tilt (persoonlijke gesl; SBT) en de waarneming van visuele verticale (subjectieve visuele verticaal, SVV) in zeven deelnemers. Omdat zowel de SVV en SBT maken gebruik van dezelfde zintuiglijke signalen, waren we in staat om een ​​7-parameter probabilistische model past bij de respons gegevens. Een van de geschatte parameters vertegenwoordigde het lawaai van de nek proprioceptoren. Dit liet ons toe om onafhankelijk te bevestigen dat de verkregen waarden voor de hals geluid overeen met die die werden gemeten in isolement. We verder gevalideerd ons model door te laten zien dat voorspellingen van ons model zijn in overeenstemming met eerder gepubliceerde tekorten in vestibulair en somatosensorische patiënten. We concluderen dat de Bayesiaanse berekeningen kan goed zijn voor de typische verschillen in ruimtelijke oriëntatie oordelen in verband met verschillende vereisten van de taak. In een vervolg op dit werk, werd deze aanpak onlangs toegepast op een patiëntenpopulatie met complete vestibulair verlies \ cite {alberts2015}. Prestaties in die patiënten vergelijkbaar met die van de controles, wat suggereert dat de sensorische gewicht was verschoven van de vestibulaire de somatische sensors.

\ Section {Oogbewegingen beïnvloeden zelf-waarneming van beweging}
Eye movement begeleiden meestal zelf-beweging om het netvlies slip minimaliseren en maximaliseren dynamische gezichtsscherpte. In \ hoofdstuk {p3} hebben we onderzocht of deze oogbewegingen hebben ook een omgekeerde rol, door te dienen als een cue voor zelf-waarneming van beweging. Om deze vraag te beantwoorden, vroegen we de deelnemers om gepercipieerde vertaling afstanden vergelijken van twee opeenvolgende, passieve, laterale hele lichaam vertalingen. Oogbewegingen tijdens deze vertalingen waren ofwel wereld-stationaire of body-stationaire. Resultaten laten zien dat de vertalingen korter werden waargenomen met body-fixed blik in vergelijking met-wereld vaste blik, wat aangeeft dat de oogbewegingen inderdaad beïnvloeden self-waarneming van beweging. Met behulp van een lineair model, schatten we de relatieve bijdrage van de vestibulaire versus de oogbeweging gebaseerd verplaatsingssignaal: de oogbeweging gebaseerd verplaatsingssignaal draagt ​​ongeveer 25 procent tot de waargenomen beweging. We onafhankelijk gevalideerd het model door het succes voorspellen van de effecten van oogbewegingen op zelf-waarneming van beweging tijdens de studies waarin de oogbewegingen waren ongedwongen. Dit toont verder dat oogbeweging signalen beïnvloeden eigen waarneming van beweging, zelfs zonder visuele stimulatie, en zelfs wanneer oculomotorische en vestibulaire schattingen in conflict, zoals tijdens-orgaan vastgestelde blik. We veronderstellen dat de nadelige gevolgen van deze schijnbaar starre opstelling minimaal zijn onder natuurlijke omstandigheden, omdat de oogbewegingen en zelf-beweging sterk gecorreleerd zijn, en omdat de oogbewegingen worden meestal begeleid door veridical optic flow aanwijzingen om zelf-beweging.

\ Section {gedeeltelijke compensatie voor fixatie diepte in zelf-waarneming van beweging}
Als oogbewegingen worden gebruikt bij de schatting van zelf-motion magnitude dient de hersenen ook de bijbehorende fixatie diepte slaan bij een veridical vertaling schatting. De reden is dat de amplitude van deze oogbewegingen, voor dezelfde fysieke vertaling, afhankelijk van de diepte van fixatie: als verre fixeren ze kleiner dan wanneer de omgeving fixeren in de wereld. In \ hoofdstuk {p4} hebben we onderzocht of de hersenen neemt inderdaad fixatie diepte te houden bij het gebruik van oogbewegingen om zelf-waarneming van beweging te vergroten. De deelnemers moesten zelf-motion tijdens verschillende omstandigheden oogbewegingen in de afwezigheid van full-field optic flow oordelen. In een 2-AFC taak, vergelijkbaar met die gebruikt in \ hoofdstuk {p3} deelnemers aangegeven of de tweede van twee opeenvolgende passieve zijdelingse gehele lichaam vertalingen langer of korter dan de eerste was. Tijdens elke vertaling, deelnemers gefixeerd ofwel een in de buurt of ver weg doelgroep, die ofwel lichaams- of world-stationaire was. Resultaten laten zien dat de waargenomen vertalingen waren korter voor nabijgelegen-wereld vaste blik in vergelijking met verre-wereld vaste blik, wat aangeeft dat de oogbewegingen niet goed geschaald in zelf-waarneming van beweging. Samen met de observatie dat zelf-waarneming van beweging niet wordt beïnvloed door de diepte van een lichaam stationaire fixatie doelgroep, kunnen we concluderen dat de oogbewegingen zijn slechts een rudimentaire cue om zelf-beweging, met een compensatie voor fixatie diepte die gedeeltelijke op zijn best.

\ Section {Gaze-afhankelijke effecten in ruimtelijke updating}
De hersenen moet ook zelf-beweging signalen naar ego-centrische ruimtelijke voorstellingen van het milieu te werken. In \ hoofdstuk {p2} onderzoeken we hoe, en in welke mate, de hersenen integreert de verschillende self-motion signalen voor de ruimtelijke-update. Deelnemers werden heen en weer geslingerd zijwaarts terwijl blik gericht op een stationaire doelwit. Wanneer de beweging richting veranderde, werd een verwijzing doel, hetzij voor of achter de fixatie punt getoond. Half een cyclus later, bij de volgende omkering, testten we bijwerking van de verwijzingen locatie door te vragen de deelnemers om te beoordelen of een kort flitste sonde werd getoond aan de linker- of rechterkant van de opgeslagen doel. De resultaten tonen aan dat zowel de richting en grootte van de bias in ruimtelijke bijwerking afhankelijk van de locatie van het object wordt geactualiseerd ten opzichte blik, hetgeen impliceert dat een blik centered referentiekader betrokken. We tonen verder aan dat deze vertekeningen kunnen worden veroorzaakt door een onderschat vertaling amplitude, een voorspanning van visueel waargenomen objecten naar de fovea, of door een combinatie van beide.

\ Section {Slotopmerkingen}

Alle experimentele hoofdstukken in dit proefschrift verslag over ruimtelijke beleving onderwerpen '. In \ hoofdstuk {p1} gebruiken we een optimale statistische integratiemodel van alle bijdragen sensorische systemen om zowel de bias en de precisie van deze waarnemingen te verklaren. In deze modellen de bijdragen van de zintuigen worden gewogen op basis van hun onzekerheid. In de hoofdstukken over laterale verplaatsing perceptie (\ hoofdstuk {p3, p4}) gebruiken we ook de wegingen van het evenwichtsorgaan en oog-beweging gebaseerde ramingen van de verplaatsing. Echter, deze weegt slechts op basis van de waargenomen biases in zelf-waarneming van beweging. Idealiter zou men dezelfde aanpak als in \ hoofdstuk {p1} te modelleren en analyseren van deze zelf-waarneming van beweging gegevens. We hebben eerste pogingen om een ​​dergelijke model gemaakt, maar verschijnen er tenminste twee belangrijke uitdagingen. Ten eerste, de geometrie van de zelf-waarneming van beweging experimenten maakt dat de eenvoudige Gauss uitkeringen van de ruimtelijke oriëntatie model te worden minder goed gedefinieerd, een scheve verdeling en we zouden moeten gebruiken roetfilter modellen om onze simulaties uit te voeren. Tweede, die modellen bevatten veel vrije parameters voor de zintuiglijke modaliteiten en priors. Met de huidige reeks experimenten we niet genoeg data om deze parameter te passen op een consistente manier te hebben. Toekomstig onderzoek moet worden gedaan om een ​​complete optimale rekening integratie van zelf-waarneming van beweging af te leiden.

In \ hoofdstuk {p2} laten we zien dat oogbewegingen uitgelokt door het fixeren van een wereld-stationaire doelwit worden in aanmerking genomen bij het updaten van de herinnerde locatie van eerder gezien doelen. Als de onderliggende signalen gebruikt voor zelf-waarneming van beweging zoals gevonden in \ hoofdstuk {p3, p4} worden ook gebruikt voor ruimtelijke bijwerken, het soort fixatie (wereld- versus body-stationaire) een effect op de waargenomen biases in ruimtelijke actualisering zou moeten hebben ook. Inderdaad, de resultaten van een proefproject in het vroege stadium van dit proefschrift suggereren dat Type fixatie beïnvloedt ruimtelijke bijwerken prestaties onder vertaling \ cite {clemens2010}. Deelnemers moesten bereiken richting herinnerde doelen na een tussenliggende vertaling. Voorlopige resultaten wijzen erop dat het bereiken van fouten zijn groot wanneer het fixeren van een body-stationaire doelwit tijdens de vertaling, terwijl er geen dergelijke fouten bestaan, terwijl het fixeren van wereldklasse stilstaande doelen \ cite {clemens2010}, wat suggereert dat inderdaad de eye-beweging gebaseerde self-motion signalen worden ook gebruikt in de ruimtelijke bijwerken.

Tot slot hoop ik nieuwe ontwikkelingen te hebben gemaakt in het begrip van de mechanismen voor ruimtelijke oriëntatie en zelf-waarneming van beweging. Natuurlijk, er zijn veel resterende vragen voor verdere studie. Deze onderzoeken moeten in de eerste plaats richten op de computationele en theoretische mechanismen, en hoe ze worden uitgevoerd door de neurale circuits. Deze studies moeten niet alleen het adres van de computationele en theoretische mechanismen, maar ook op de neurale implementatie en routes die kunnen worden gevonden in de hersenen.
