\clearpage
%\pagestyle{empty}

\chapter*{Nederlandse samenvatting}
\phantomsection
\addcontentsline{toc}{chapter}{Nederlandse samenvatting}

Wanneer men door de omgeving navigeert, dan veranderen de visuele, auditieve, vestibulaire, tactiele, en motorische signalen die de hersenen bereiken. Ondanks deze, door beweging veroorzaakte veranderingen, nemen we de wereld als een stabiele werkelijkheid waar. We onderhouden een gevoel van waar we zijn, wat onze ori\"entatie in de wereld is, en interacteren ogenschijnlijk moeiteloos met de objecten om ons heen. De processen die hier aan te grondslag liggen zijn het onderwerp van dit proefschrift. Met behulp van computationele modellen en psychometrische experimenten leggen we de fysische en biologische beperkingen in de interactie tussen het evenwichtsorgaan en andere zintuiglijke systemen voor ruimtelijke ori\"entatie en de waarneming van zelfbeweging bloot. Dit onderzoek heeft geresulteerd in de volgende bijdragen aan het onderzoeksveld:

\begin{description}
\item[\Hoofdstuk{p1}] Spati\"ele ori\"entatie kan beschreven worden middels statistisch optimale integratie van somatosensorische, proprioceptieve, en vestibulaire signalen. Hoewel deze zintuiglijke systemen moeilijk afzonderlijk te bestuderen zijn, hebben we hun statistische eigenschappen kunnen afleiden door statistische optimaliteit als uitgangspunt te nemen om het gedrag te verklaren. De gevonden statistische eigenschappen, in combinatie met de veronderstelde architectuur van het systeem hebben we vervolgens gebruikt om de klinische afwijkingen in twee pati\"entgroepen te verklaren.

\item[\Hoofdstuk{p3}] De perceptie van zelfbeweging wordt beïnvloed door oculomotorische signalen. Dit gebeurt zelfs wanneer de zelfverplaatsing afgeleid uit de oogbewegingen in strijd is met de vestibulair waargenomen zelfbeweging en in afwezigheid van een visuele stimulus.

\item[\Hoofdstuk{p4}] Deze oculomotorische signalen zijn slechts een rudimentaire bron voor de waarneming van zelfbeweging. Als de fixatieafstand toeneemt, worden de oogbewegingen kleiner voor dezelfde zelfbeweging. Deze diepteschaling wordt echter niet voldoende gecompenseerd in de perceptie van zelfbeweging.

\item[\Hoofdstuk{p2}] Zelfbewegingssignalen beïnvloeden de dynamische perceptie van de wereld om ons heen. Fouten die ontstaan tijdens het updaten van objectlocaties tijdens zelfbeweging suggereren dat hiervoor oogcentrische referentie kaders worden gebruikt.
\end{description}

In de volgende paragrafen, zullen we elk van deze bijdragen in meer detail bespreken.

\section{Multisensorische verwerking voor ruimtelijke ori\"entatie}

Vele studies hebben aangetoond dat de hersenen de signalen van verschillende zintuigen op een statistisch optimale wijze integreren om tot een percept te komen. Normaliter wordt deze statistisch optimale integratie aangetoond door de ruisniveaus van de afzonderlijke modaliteiten te bepalen en op basis hiervan te voorspellen wat het ruisniveau van alle modaliteiten in combinatie is. In ruimtelijke ori\"entatie is het echter moeilijk om zintuiglijke signalen afzonderlijk te meten, vooral omdat het vestibulaire systeem niet of moeilijk uit te schakelen is bij het meten van somatosensorische bijdragen.

In \Hoofdstuk{p1} hebben we statistische optimaliteit als uitgangspunt genomen, en op die manier de perceptuele verschillen tussen de waarneming van “het hoofd in de ruimte” ten opzichte van “het lichaam in de ruimte” verklaard Met behulp van een psychofysische aanpak hebben we zowel de waarneming van lichaamshoek (Subjective Body Tilt; SBT) als de waarneming van de visuele verticaal (Subjective Visual Vertical; SVV) in zeven proefpersonen gemeten. Omdat zowel de SVV als de SBT gebruik maken van dezelfde zintuiglijke signalen, waren we instaat om een model met 7-parameters te fitten op de verkregen data. Een van de voorspelde parameters was de ruis in de nekproprioceptie. Omdat de ruis in de nekproprioceptie wel in afzondering te meten is, hebben we de modelwaarden kunnen bevestigen. Daarnaast hebben we het model gevalideerd door te laten zien dat voorspellingen over vestibulaire en somatosensorische pati\"enten in overeenstemming zijn met gepubliceerde gegevens. We concluderen dat een Bayesiaanse aanpak de typische taakafhankelijke verschillen in de waarneming van ruimtelijke ori\"entatie goed kan verklaren. In een vervolg op dit werk werd deze aanpak onlangs toegepast op een pati\"entenpopulatie met verlies van het complete vestibulaire systeem \cite{alberts2015}. Deze pati\"enten voerden de taak vergelijkbaar met de controle populatie uit, wat de suggestie wekt dat de sensorische gewichten verschoven zijn van vestibulaire naar somatische zintuigen.


\section{Oogbewegingen beïnvloeden de waarneming van zelfbeweging}

Tijdens zelfbeweging maken we meestal oogbewegingen om te compenseren voor verstoringen van de visuele beelden op het netvlies. In \Hoofdstuk{p3} onderzochten we of deze oogbewegingen ook een omgekeerde rol hebben, dat wil zeggen of ze ook gebruikt worden als bronsignaal voor zelfbeweging. Om een antwoord op deze vraag te geven, vroegen we proefpersonen om de bewogen afstand te vergelijken in twee opeenvolgende passieve zijwaartse verplaatsingen. De ogen keken tijdens deze verplaatsingen naar een wereld- of lichaamsvast doel. Onze resultaten laten zien dat bewegingen als korter werden waargenomen wanneer naar een lichaamsvast doel werd gekeken, en de oogbewegingen dus werden onderdrukt. Met behulp van een lineair model hebben we de relatieve bijdrage van het op oogbewegingen gebaseerde verplaatsingssignaal ten opzichte van de vestibulaire bijdrage bepaald. De geschatte zelfbeweging gebaseerd op de oogbewegingen wordt voor ongeveer 25 procent meegenomen in de waarneming van de zelfbeweging. Het model is onafhankelijk gevalideerd door de effecten van oogbewegingen op de waarneming van zelfbeweging zonder een visueel fixatiedoel te voorspellen en te vergelijken met psychofysische data. Dit laat zien dat oogbewegingen zelfs gebruik worden voor het waarnemen van zelfbeweging zonder visuele stimulatie, en wanneer de oculomotorische en vestibulaire schattingen met elkaar in conflict zijn, zoals tijdens lichaamsvaste fixatie.

Onder natuurlijke omstandigheden worden vaak wereldvaste doelen gefixeerd, waardoor  de oogbewegingen sterk gecorreleerd zijn met zelfbeweging. Daarom nemen we aan dat de nadelige gevolgen van deze inflexibele integratie strategie in het dagelijks leven minimaal zijn. Daarnaast zorgt zelfbeweging voor verplaatsing van de beelden op de retinae, deze verplaatsingen worden onder natuurlijke omstandigheden ook gebruikt voor de schatting van zelfbeweging, die de relatieve bijdrage van de oogbewegingssignalen beperkt.


\section{Fixatiediepte wordt slechts gedeeltelijk gecompenseerd voor de waarneming van zelfbeweging}

Als oogbewegingen inderdaad gebruikt worden voor de waarneming van zelfbeweging, dan moeten de hersenen deze oogbewegingen schalen met de fixatiediepte om tot een correcte schatting van zelfbeweging te komen. Dit komt doordat de amplitude van de oogbewegingen, bij gelijkblijvende lichaamsverplaatsing, afhangt van de fixatiediepte: als men naar een doel heel ver weg kijkt dan bewegen de ogen nauwelijks, terwijl ze tijdens een nabije fixatie meer bewegen. In \Hoofdstuk{p4} hebben we onderzocht of de hersenen fixatiediepte inderdaad verdisconteren in de op oogbeweging bepaalde waarneming van zelfbeweging. In een 2-AFC taak, vergelijkbaar met die gebruikt is in \Hoofdstuk{p3}, moesten proefpersonen aangeven of de tweede van twee opeenvolgende passieve bewegingen langer of korter was dan de eerste. Tijdens deze bewegingen moesten de proefpersonen op een dichtbij of veraf doel fixeren, dat vast stond ten opzichte van de wereld of het lichaam. Onze resultaten laten zien dat bewegingen als korter werden waargenomen naarmate wereldvaste fixatiepunten dichterbij kwamen te liggen. Dit effect verdween echter wanneer proefpersonen naar een lichaamsvast fixatiepunt keken. Uit deze bevindingen concluderen we dat oogbewegingen slechts rudimentaire informatie bieden over zelfbeweging, waarbij de compensatie voor fixatie diepte slechts gedeeltelijk is.


\section{Oogcentrische effecten in het bijwerken van spati\"ele informatie}

De zelfbewegingssignalen uit de \Hoofdstuk{p3,p4} worden ook gebruikt om egocentrische representaties van de omgeving bij te werken. In \Hoofdstuk{p2} onderzoeken we hoe, en in welke mate, de hersenen de verschillende zelfbewegingssignalen integreren om representaties van de omgeving te corrigeren voor zelfbeweging. Proefpersonen werden zijwaarts heen-en-weer bewogen terwijl ze een doel fixeerden dat vast stond in de wereld. Wanneer de bewegingsrichting veranderde, werd een lampje voor of achter het fixatiepunt geflitst. De proefpersonen werd gevraagd de positie van dit lampje te onthouden. De relatieve positie van de proefpersoon ten opzichte van het lampje verandert echter wanneer hij van de ene naar de andere kant beweegt. Een halve cyclus na de eerste flits, testten we of deze zelfbeweging correct is verdisconteerd door een ander lampje links of recht van het onthouden lampje te flitsen. De resultaten laten zien dat zowel de richting als de grootte van de correctiefout afhangt van de locatie van het object relatief ten opzichte van het fixatiepunt. Dit suggereert dat het object relatief ten opzichte van het fixatiepunt wordt opgeslagen door de hersenen. Verder laten we zien dat de fout kan worden verklaard door zowel een onderschatting van de bewogen afstand, als door de aanname (i.e. een Bayesiaanse prior) dat visuele objecten worden waargenomen op de fovea.


\section{Slotopmerkingen}

De experimentele hoofdstukken in dit proefschrift hebben het ruimtelijke waarnemen van proefpersonen als gemeenschappelijk thema. In \Hoofdstuk{p1} hebben een statistisch optimaal integratiemodel gebruikt om zowel de accuraatheid als de precisie van deze ruimtelijke waarnemingen verklaren. In dit soort modellen worden de verschillende zintuiglijke signalen gewogen aan de hand van de onzekerheid in het signaal. In de hoofdstukken over de waarneming van zijwaartse verplaatsing (\Hoofdstuk{p3,p4}) hebben we ook gebruik gemaakt van een gewogen combinatie van zintuiglijke signalen, maar deze zijn niet gebaseerd op de gemeten of afgeleide onzekerheid van de individuele signalen. Idealiter zou men dezelfde insteek gebruiken als in \Hoofdstuk{p1} om de resultaten uit deze hoofdstukken te analyseren. We hebben de eerste stappen genomen om een dergelijk model te implementeren, maar er zijn hierbij tenminste twee grote uitdagingen te overwinnen. Ten eerste leidt de geometrie van onze zelfbewegingsexperimenten tot scheve, niet Gaussiaanse, kansverdelingen. Dit maakt dat standaardtechnieken die gebruikt worden bij normale kansverdelingen onbruikbaar zijn, waardoor we aangewezen zijn op bijvoorbeeld sequenti\"ele Monte Carlo methoden. Daarnaast is onze dataset te beperkt om alle vrije parameters uit een kansmodel van onze taken op een consistente manier te schatten. Het ontwikkelen van een compleet optimaal integratiemodel voor het waarnemen van zelfbeweging is een logische, maar niet triviale, vervolgstap in dit onderzoek.

In \Hoofdstuk{p2} laten we zien dat oogbewegingen die veroorzaakt worden door het fixeren van een wereldvast fixatiepunt gebruikt worden om de herinnerde locatie van doelen in de omgeving bij te werken. Als het onderliggende signaal dat gebruikt wordt voor de waarneming van zelfbeweging (zie \Hoofdstuk{p3,p4}) ook gebruikt wordt voor het bijwerken van onthouden spati\"ele doelen, dan zou het soort fixatiepunt (wereld- of lichaamsvast) een effect moeten hebben op de fout in het bijwerken van deze doelen. Resultaten van een eerder pilot-onderzoek suggereren dat het soort fixatie inderdaad een dergelijk effect laat zien \cite{clemens2010}. We lieten proefpersonen na een zijwaartse beweging reiken naar herinnerde doelen. De eerste resultaten laten zien dat fouten in reikbewegingen groot zijn wanneer een lichaamsvast doel wordt gefixeerd, terwijl er geen fouten optreden bij de fixatie van een wereldvast doel \cite{clemens2010}. De voorlopige conclusie is dat oogbewegingen inderdaad gebruikt worden om zelfbeweging te schatten voor het bijwerken van onthouden doelen.

Tenslotte, ik hoop met dit proefschrift een bijdrage geleverd te hebben aan een beter begrip van de mechanismen voor ruimtelijke ori\"entatie en de waarneming van zelfbewegingen. Natuurlijk laat mijn proefschrift veel vragen liggen voor vervolgonderzoek, en roept nieuwe vragen op. Deze onderzoeken zouden zich niet alleen op de computationele en theoretische mechanismen moeten richten, maar ook op de implementatie van deze mechanismen in het brein.
