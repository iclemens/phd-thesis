\section{Discussion}

We investigated the contribution of eye movements to the perception of passively-induced self-motion. Experiments were performed in the absence of full-field optic flow to eliminate the contribution of this visual motion signal. Perception of self-motion was compared across three fixation types: during free fixation the fixation target was extinguished before the movement, while during world and body fixation targets remained stable relative to the world and body, respectively. Our results show that self-motion is underestimated during body fixation (in which the eyes remain stationary) compared to world fixation (in which the eyes move to maintain fixation). To characterize the separate vestibular and eye movement contributions, we fitted a single parameter model to the perceptual responses for the body versus world comparison conditions and validated this model independently by predicting the effects of eye movements on self-motion perception during free fixation conditions. This model takes into account subject specific oculomotor weight and eye movement patterns. Based on these inputs it accurately predicts the responses in the free fixation conditions. This demonstrates that extra-retinal eye movement signals are used as a cue in the perception of self-motion, contributing significantly to the self-motion percept with a weight of approximately 25 percent, even in the absence of optic flow.

It is surprising that an influence of eye movements can be observed even for body- stationary fixations, during which the stationary eye movement signal is clearly in conflict with the non-zero vestibular signal. While this demonstrates the strength of the assumption that fixation targets are world-stationary, it raises the question how reliable this assumption is. Simultaneous recording of angular head and eye movements during natural behavior reveals that approximately 80 percent of eye movements can be classified as compensatory, i.e. eye movements directed opposite to head movement and therefore consistent with maintenance of world-fixed fixation \cite{einhauser2007}. Similarly, other studies have shown that world-stationary fixations are common for many every day activities, ranging from making a cup of tea \cite{hayhoe2014} to driving a car \cite{land1994}, to walking \cite{foulsham2011} and even reaching, where people tend to look at the source and destination of the object, but not at the hand \cite{flanagan2003}. Because world-stationary fixations are so common, the natural world statistics imply that self-motion and eye movements are highly correlated, thus making eye movements a fairly reliable cue for self-motion.

Even when fixation is not world-fixed, eye movement signals are combined with optic flow signals to yield realistic self-motion estimates (e.g. \cite{royden1992,vandenberg2000}). During world-fixed fixation, the eyes move to compensate for body translation, thereby reducing the optic flow component in the retinal signal. The self-motion estimate will therefore be driven predominantly by the eye movement signal. On the other hand, in the body-fixed condition, eye movements are minimal and optic flow maximal such that perceived self-motion will be driven predominantly by the optic flow signal itself. Because our experiment was performed in darkness, this optic flow signal was absent in the body-fixed condition which can explain why self-motion was underestimated.

During body and world fixation, eye movements are driven by retinal slip of the fixation target. However, in the free fixation condition, retinal slip is not available and resulting eye movements resemble the linear vestibulo-ocular reflex (LVOR), in that the gain relative to world fixation was ~0.4 (see Figure 3B; \cite{ramat2003}). This reflex is thought to be driven by a double integration of the vestibular signal, converting the head acceleration signal from the otoliths to eye position \cite{green2007,walker2010}. If eye movements during free fixation are in fact vestibularly driven, then combination of this eye movement signal with the vestibular signal itself seems redundant. However such combination could reflect a strategy to reduce noise. Both the direct (vestibular) and indirect (LVOR) signals depend on integration of the linear acceleration signal and may be corrupted by independent noise sources. Combining them in a statistically optimal fashion will decrease the noise level towards the noise level of the original source signal \cite{faisal2008,clemens2011,fetsch2013}. The consequence of this integration will be a reduced self-motion estimate when the gain of the LVOR is less than 1, as we observed in the free condition.
Alternative interpretations

In the above, we assumed that the eye movements themselves drive perception of self-motion. However, it is conceivable that a common correlate of eye movements, such as attention or visual motion influenced our results. In 1963, Guedry and Harris reported a substantial underestimation of displacement when their observers watched a small body-fixed target compared to displacements in the dark. However, they attributed their findings to an attentional shift from judgments of body displacement in the dark to judgments of target displacement in the fixation condition. Although our results are in line with their observations, our eye movement analysis as well as the relation to perceptual judgments suggest that eye movements modulate self-motion perception, possibly in addition to attentional factors \cite{kitazaki2003}.

Others have reported errors in the disambiguation of self and object-motion. Examples include the perceived motion of body-fixed visual targets during angular acceleration (the oculogyral illusion; \cite{carriot2011}), the apparent displacement of body-fixed stimuli during linear acceleration (the oculogravic illusion; \cite{graybiel1952}) and the apparent movement of world-stationary targets during self-motion in darkness \cite{dyde2008}. Similar disambiguation errors could cause the effects we observed. More specifically, if movement of the fixation point relative to the observer were always attributed to self-motion, then self-motion would be underestimated during body relative to world fixation, as we observed. However, such attribution errors cannot account for the effects in the free condition, because no fixation point was visible and no attribution was required. In the free condition, we demonstrate that eye movements by themselves, occurring in the absence of visual tracking and other external cues, influence the perception of self-motion.

\subsection{Implications for other studies}

Many previous self-motion studies have used a body-fixed fixation point to control for eye movement related effects. Our results suggest, however, that using a body-fixed fixation point causes underestimation of self-motion. For example, Li, Wei, and Angelaki \citeyear{li2005a} (2005) investigated spatial updating across lateral translation and found that saccades to updated targets undershot the actual target location. As self-motion perception drives this update, the effects of eye movements on self-motion perception should also influence the updating process. In other words, the observed undershoot could be due to the underestimation of self-motion caused by the body-fixed fixation point. Another example is a study on the perception of vertical object-motion during lateral translation \cite{dokka2013}. This study reports incomplete compensation for self-motion when judging the deviation from vertical motion of a moving object. This observation could also be due to underestimation of self-motion induced by the fixation of the body-fixed target. 

A moving fixation point is also known to influence self-motion perception, as in the Slalom Illusion \cite{freeman2000}; observers viewing expanding optic flow while fixating a target that oscillates from left to right perceive slaloming motion which is inconsistent with the purely forward motion specified by the expanding optic flow display. However, this observation is consistent with the idea that oculomotor signals are used in estimating self-motion. Additionally, it has been shown that eye movements affect postural sway \cite{glasauer2005}.  Participants performed smooth pursuit eye movements in complete darkness and displayed lateral sway consistent with the stabilization of posture using a self-motion estimate influenced by pursuit eye movements.

Studies conducted to characterize vestibular-only sensitivity are often performed in complete darkness or with closed eyes \cite{grabherr2008,macneilage2010b,macneilage2010a,roditi2012,valko2012,nesti2014}. However, the results of our free-fixation condition suggest that even under these circumstances, results could easily be influenced by vestibularly driven eye movements. Overall, we suggest that any study concerned with self-motion processing must consider the possible influence of eye movements.

\subsection{Possible neural substrate}

This leaves us with the question of where in the brain these effects originate. The locus of our effect is likely to carry both eye movement and vestibular signals. Prime candidate areas known to carry both vestibular and eye movement signals are the vestibular nuclei \cite{henn1975,daunton1979} and the cerebellum \cite{waespe1981}. On the other hand, eye movements could influence self-motion perception indirectly via optic flow processing. In particular, cortical areas that carry both vestibular and optic flow signals (which can be modulated by eye movements) include the ventral intraparietal area (VIP; \cite{bremmer2002,chen2011}), and the dorsal medial superior temporal area (MSTd; \cite{gu2008}). Future work should reveal how such brain areas, directly or indirectly, merge both vestibular and oculomotor signals into a coherent percept of self-motion.
