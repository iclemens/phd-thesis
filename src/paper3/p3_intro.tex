\section{Introduction}

An accurate estimate of self-motion is important to guide interactions with the environment. During passive self-motion both vestibular and optic flow signals provide information about self-motion \cite{gibson1955, benson1986, harris2000, israel1989, angelaki2005, carriot2013, chen2010}. However, also compensatory eye movements that maintain fixation on world-fixed objects carry self-motion information. These eye movements are driven by retinal slip or vestibular signals. For example, the linear vestibulo-ocular reflex (LVOR) stabilizes gaze during head translations, even in complete darkness \cite{paige1989,medendorp2002,angelaki2004}.  Many have shown that the brain uses oculomotor signals to extract the optic flow component related to self-motion \cite{warren1988, royden1992, freeman1998, lappe1999}, but to our knowledge a possible direct influence of eye movements on self-motion perception has not been investigated. Here we investigate whether these oculomotor signals are also used to estimate self-motion directly.

When gaze is world-stable during whole-body translation, the eye displacement correlates with translation size and is modulated by fixation depth \cite{schwarz1989, paige1998, mchenry2000, medendorp2002}. When properly scaled this eye movement signal could serve as a self-motion cue. In contrast, when fixation is body-fixed the eyes remain stationary in their orbits \cite{paige1998, ramat2005} making them no longer informative about self-motion. If, however, the brain assumes that eye movements are always made to maintain world-stable gaze, as in the LVOR, it would equate the absence of eye movements with the absence of self-motion. As a result, self-motion with body-fixed gaze should be underestimated compared to self-motion with world-fixed gaze, despite identical vestibular cues.

 This hypothesis implies that oculomotor signals are always combined with vestibular signals to estimate self-motion, even in complete darkness.  In this case, the size of the unconstrained eye movements should resemble a VOR movement that is intermediate between body- and world-fixed fixation, and should parametrically relate to the perceived self-motion.

To test whether eye movements are used in self-motion perception, we employed a two-alternative forced choice (2-AFC) paradigm in which participants were presented with two consecutive lateral translations. They had to indicate whether the second translation was longer or shorter than the first. Eye movements during each interval were either constrained using a  body or world stationary fixation point or not constrained at all (i.e. free). We show that identical translations were perceived shorter when gaze was body stationary compared to world stationary. Furthermore, using a linear model we predicted perceived displacement during the free gaze condition based on vestibular signals and unconstrained eye movements. We conclude that eye movements influence self-motion perception even in the absence of optic flow or other visual stimulation. 