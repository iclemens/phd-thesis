\section{Discussion}

[Introduction]

Discuss how VOR scaling relates to our results!

We compared the parameter for the near condition, , found in the present paper to the one we previously found in a similar experiment where we compared body-fixed to world-fixed fixations at near (50 cm) distances only. Figure 6 shows how well our model explains the data in our previous paper, the positive correlation between the actual PSEs and those predicted using the model in this paper (stats) adds confidence to the parameter values for  presented here. The average difference between the values found here and those reported previously (see Table 2) is 12 ±8 percent-points, indicating a variation of about 12\% on the contribution of the vestibular system vs that of the visual system between the present and our previous paper. While the variation might seem high, keep in mind that the two parameter values presented in this study are fitted on only four conditions, which do not overlap with the two conditions used to fit  parameter  previously.
There are two possible explanations for our results, first the weight could be a function of fixation depth; second it could be that eye movements are used to modulate self-motion perception.

If the eye movement weight was a function of fixation depth, and the oculomotor and vestibular signals are combined in a statistically optimal fashion, the vestibular weight would we weighted more with increasing fixation depth. As there is no effect in the body condition, we can rule out that fixation depth is taken into account in a statistically optimal fashion.

Should eye movements influence self-motion perception, then we would expect no effects in the body near versus body far condition because of the absence of eye movements in this condition. As the eye movements in the world near interval are smaller than those in equidistant world far intervals, we would expect underestimation of self-motion in the world far condition when compared to the world near condition. Note that it is possible that some compensation for the effects of depth on the eye movements is present, but as long as such compensation is incomplete effects between the world fixed condtions are still to be expected.

[Conclusion]