\section{Introduction}

An accurate estimate of self-motion is important when navigating through our environment. During navigation, we typically keep track of the objects we interact with or do not want to bump into using eye movements. The shift of the retinal image caused by self-motion (i.e., optic flow) is disturbed by these eye movements. However, using the oculomotor signal, the brain accounts for these disturbances by internally separating optic flow in two components, one caused by self-motion and the other by eye movement (references).

When the eyes track world-centered objects, their angular displacement is directly related to the size of the motion. Therefore, we recently proposed that these tracking eye movements could also be used as a self-motion cue, in addition to optic flow and vestibular signals. To test this hypothesis, we compared self-motion perception in the absence of full-field optic flow during passively induced whole-body translations \cite{clemens2015a}. Supporting this notion, our results showed that self-motion is underestimated during body-centered fixations (in which the eyes remain stationary in their orbits) compared to when world-centered fixations (in which the eyes must move to maintain fixation).

The eye movements that keep fixation on a world-centerd target while being translated, e.g. those driven by the VOR, must scale with fixation depth. For body-centered targets these eye movements must be suppressed instead. Conversely, to serve as an adequate self-motion cue, the brain must internally scale the ensuing eye movement by fixation distance. Because we did not manipulate fixation distance in our previous study, we cannot infer whether eye movements are used as-is as a rudimentary cue for self-motion, or are properly scaled before influencing self-motion perception.

In the present study, we investigate how fixation distance influences perception of self-motion across passive lateral translation. Using a psychophysical approach, participants had to indicate whether the second body displacement of two one-second translation intervals was smaller or longer than the first. We show that translation amplitude is perceived smaller when fixating a far rather than a nearby world-centered target, indicating that eye movements are not properly scaled for self-motion perception. Together with the further observation that self-motion perception with body-centered fixation targets is not affected by fixation depth, we conclude that eye movements are merely a rudimentary cue to self-motion, with a compensation for fixation depth that is partial at best.