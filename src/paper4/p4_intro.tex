\section{Introduction}

Our internal estimate of self-motion is used in a multitude of ways, among others, it keeps us from bumping into objects by providing accurate feedback on our movements and it allows us to track objects in the environment across intermittent self-motion. The vestibular system as well as optic flow provide essential information about self-motion \cite{gibson1955, benson1986, harris2000, israel1989, angelaki2005, carriot2013, chen2010}. However, the eye movements that typically occur to maintain visual acuity on important objects in the environment disturb the optic flow signal. Using the oculomotor signal, the brain accounts for these disturbances by internally separating optic flow in two components, one caused by self-motion and the other by eye movement  \cite{warren1988, royden1992, freeman1998, lappe1999}.

When the eyes track world-centered objects, their angular displacement is directly related to the size of the motion of the observer \cite{schwarz1989, paige1998, mchenry2000, medendorp2002}. Because  the majority of fixations are on world-stationary  objects, we recently proposed that these tracking eye movements could also be used as a self-motion cue, in addition to optic flow and vestibular signals. To test this hypothesis, we compared self-motion perception in the absence of full-field optic flow during passively induced whole-body translations \cite{clemens2015a}. Supporting the notion that eye movement can act as a self-motion cue, our results showed that self-motion is underestimated during body-centered fixations (in which the eyes remain stationary in their orbits) compared to fixations on world-stationary objects (in which the eyes must move to maintain fixation).

Geometrically, eye movements that keep fixation on a world-centered target during lateral whole body translation (i.e. the linear vestibulo-ocular reflex; LVOR), must scale with fixation depth \cite{angelaki2004}. When fixating body-centered targets these eye movements must be suppressed irrespective of fixation distance \cite{angelaki2004}. Conversely, to serve as an adequate self-motion cue, the brain must internally scale the ensuing eye movement by fixation distance. Because we did not manipulate fixation distance in our previous study, it is unknown whether eye movements are used as a rudimentary cue for self-motion (i.e. without taking fixation depth into account), or are properly scaled in the mechanisms for self-motion perception.

In the present study, we investigate how fixation distance influences perception of self-motion during passive lateral translation. Using a psychophysical approach, participants had to indicate whether the second body displacement of two one-second translation intervals was smaller or longer than the first. We show that translation amplitude is perceived smaller when fixating a far compared to a nearby world-centered target, indicating that eye movements are not properly scaled in self-motion perception. Together with the observation that self-motion perception is not affected by the depth of a body-centered fixation target, we conclude that eye movements are merely a rudimentary cue to self-motion, with a compensation for fixation depth that is partial at best.
